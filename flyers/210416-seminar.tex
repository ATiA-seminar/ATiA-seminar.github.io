% !TeX program = xelatex
\documentclass{UAmathtalk}

\author{Luis Scoccola}
\address{Michigan State University}
\urladdr{https://luisscoccola.github.io/}
\title{Approximate and Discrete Vector Bundles in Theory and Applications}
\date{Friday, April 16th, 2021}


\begin{document}

\maketitle

\begin{abstract}
Synchronization problems, such as the problem of reconstructing a 3D shape from a set of 2D projections, can often be modeled by principal bundles. Similarly, the application of local PCA to a point cloud concentrated around a manifold approximates the tangent bundle of the manifold. In the first case, the characteristic classes of the bundle provide obstructions to global synchronization, while, in the second case, they provide topological information of the manifold beyond its homology, and in particular, give obstructions to dimensionality reduction. I will describe joint work with Jose Perea in which we propose several notions of approximate and discrete vector bundle, study the extent to which they determine true vector bundles, and give algorithms for the stable and consistent computation of low-dimensional characteristic classes directly from these combinatorial representations. No previous knowledge of the theory of vector bundles will be assumed.
\end{abstract}

\end{document}
