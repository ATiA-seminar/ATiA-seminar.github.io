% !TeX program = xelatex
\documentclass{UAmathtalk}

\author{Georg Osang}
\address{IST Austria}
\urladdr{https://dblp.org/pid/73/10826.html}
\title{Order-$k$ Delaunay Mosaics and Alpha Shapes}
\date{Friday, March 12, 2021}


\begin{document}

\maketitle

%\begin{center}
%\Large{Workshop in Topology: Identifying Order in Complex Systems}
%
%\normalsize{(This meeting is unofficially partnered with ATiA and meets via a different link.)}
%\end{center}

\begin{abstract}
Given a finite point set in a $d$-dimensional Euclidean space, order-$k$ Delaunay mosaics, order-$k$ Voronoi tessellations and order-$k$ $\alpha$-shapes are generalizations of Delaunay triangulations, Voronoi tessellations and $\alpha$-shapes, respectively.
We introduce a $(d+1)$-dimensional geometric cell complex, the rhomboid tiling, which is dual to a well-known hyperplane arrangement, and present properties and relationships between each of the aforementioned notions.
Insights about the rhomboid tiling give us a simple and efficient algorithm to compute order-$k$ Delaunay mosaics, and by extension order-$k$ $\alpha$-shapes which are subcomplexes of the order-$k$ Delaunay mosaic.
$\alpha$-shapes were originally introduced as cell complexes capturing the ``shape'' of a finite point set and later to compute persistent homology. Their order-$k$ generalizations are more robust against noise, however, this comes at a cost of higher complexity. We explore this increased complexity experimentally using our open-source implementation of the algorithm for order-$k$ Delaunay mosaics. This is joint work with Herbert Edelsbrunner.
\end{abstract}

\end{document}
