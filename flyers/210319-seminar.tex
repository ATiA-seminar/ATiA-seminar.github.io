% !TeX program = xelatex
\documentclass{UAmathtalk}

\author{Ren\'{e} Corbet}
\address{KTH}
\urladdr{https://www.kth.se/profile/corbet?l=en}
\title{Computing the Multicover Bifiltration}
\date{Friday, March 19, 2021}


\begin{document}

\maketitle

%\begin{center}
%\Large{Workshop in Topology: Identifying Order in Complex Systems}
%
%\normalsize{(This meeting is unofficially partnered with ATiA and meets via a different link.)}
%\end{center}

\begin{abstract}
Given a finite set $A\subset\mathbb{R}^d$, let $\textnormal{Cov}_{r,k}$ denote the set of all points within distance $r$ to at least $k$ points of $A$. Allowing $r$ and $k$ to vary, we obtain a 2-parameter family of spaces that grow larger when $r$ increases or $k$ decreases, called the \emph{multicover bifiltration}. Motivated by the problem of computing the homology of this bifiltration, we introduce two closely related combinatorial bifiltrations, one simplicial and the other polyhedral, which are both topologically equivalent to the multicover bifiltration and far smaller than a \v Cech-based model considered in prior work of Sheehy. Our polyhedral construction is a variant of the \emph{rhomboid tiling} of Edelsbrunner and Osang, and can be efficiently computed using a variant of an algorithm given by these authors. Our simplicial construction is conceptually simpler and more general. Furthermore, it is useful for understanding the polyhedral construction, and proving its correctness. If time permits, we will also have a look at experimental results. \\
This is joint work with Michael Kerber, Michael Lesnick, and Georg Osang.
\end{abstract}

\end{document}
